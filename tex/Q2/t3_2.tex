\documentclass{article}%
\usepackage{import}
\subimport{../}{packages}

\begin{document}


\begin{problem}
Случайные величины $X$ и $Y$ независимы и имеют равномерное
распределение на отрезке $[<a>;<b>]$. Для случайной величины $Z=X+Y$ найдите:
1) функцию распределения $F_Z(x)$;
2) плотность распределения $f_Z(x)$ и постройте график плотности;
3) вероятность $\P(<d>\leqslant Z\leqslant<c>)$.
\end{problem}

\begin{solution*}
1) Функция распределения $F_Z(x)$ имеет вид:
$
F_Z(x)=\left\{
\begin{array}{l}
0, x\leqslant <Q0>;\\
<Q3>, <Q0>\leqslant x\leqslant <Q1>;\\
<Q4>, <Q1>\leqslant x\leqslant <Q2>;\\
1, x\geqslant <Q2>;
\end{array}.
\right.
$
2) Плотность распределения $f_Z(x)$ имеет вид:
$
f_Z(x)=\left\{
\begin{array}{l}
<Q5>, <Q0>\leqslant x\leqslant <Q1>;\\
<Q6>, <Q1>\leqslant x\leqslant <Q2>;\\
0, x\not\in [<Q0>;<Q2>];
\end{array}.
\right.
$

3) Вероятность равна:
$
\P(<d>\leqslant Z\leqslant<c>)=<P>.
$
\end{solution*}


\end{document}